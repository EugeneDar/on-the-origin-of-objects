\documentclass[sigplan,11pt,nonacm,natbib=false]{acmart}
\settopmatter{printfolios=false,printccs=false,printacmref=false}
\usepackage[maxnames=1,minnames=1,maxbibnames=100,natbib=true,citestyle=authoryear,bibstyle=authoryear,doi=false,url=false,isbn=false,isbn=false,backend=biber]{biblatex}
\addbibresource{main.bib}
\ifnum\pdfshellescape=1
  \usepackage[finalizecache]{minted}
\else
  \usepackage[frozencache]{minted}
\fi
\usepackage{ffcode}
\usepackage[capitalize]{cleveref}
\usepackage{to-be-determined}

\usepackage{csquotes}
\usepackage[T2A]{fontenc}
\usepackage[russian,english]{babel}
\usepackage{substitutefont}
  \substitutefont{T2A}{LinuxBiolinumT-TLF}{LibertinusSans-TLF}
  \renewcommand\ttdefault{cmtt}

\setlength{\footskip}{13.0pt}

\title{On the Origins of Objects by Means of Careful Selection}
\subtitle{%
  Ver:
  \texorpdfstring{
    \href{https://github.com/REPOSITORY/releases/tag/0.0.0}
      {\ff{0.0.0}}
  }{0.0.0}
}
\author{Yegor Bugayenko}
\orcid{0000-0001-6370-0678}
\email{yegor256@gmail.com}
\affiliation{
  \institution{Huawei}
  \city{Moscow}
  \country{Russia}
}
\ccsdesc[100]{Object-Oriented Programming}
\keywords{Object-Oriented Programming}

\newcommand\aff[1]{\ff{\textcolor{gray}{$\star$}#1}}
\newcommand\deff[1]{\ff{\textcolor{blue!50!black}{\textbf{#1}}}}
\newcommand\adeff[1]{\aff{\textcolor{blue!50!black}{\textbf{#1}}}}
\newcommand\eohex[1]{\ff{#1}}

\begin{document}
\raggedbottom

\begin{abstract}
We introduce a taxonomy of objects for EO programming language. This taxonomy is designed with a few principles in mind: non-redundancy, simplicity, and so on. The taxonomy is supposed to be used as a navigation map by EO programmers. It may also be helpful as a guideline for designers of other object-oriented languages or libraries for them.
\end{abstract}

\maketitle

\section{Introduction}

Each object-oriented programming language offers its own set of objects, classes, types, functions, constants, traits, templates, and so on, which programmers can use off-the-shelf to model their specific use cases:
``Development Kit'' in Java,
``Standard Library'' in C++, Python, Swift, Rust, and Ruby,
``.NET Standard'' in C\#,
``Built-in Objects'' in JavaScript,
and so on.

Each Standard Library (SL) freely defines its own principles of abstraction. For example, to get an absolute value of a numeric object \ff{x} in Java, one has to call a static method of another class \ff{Math.abs(x)}, while in Ruby it would be a property of the same object \ff{x.abs}. For example, in Java, an attempt to get a value from a hash map by a key will produce \ff{null} in case of its absence, while in C++ it will return an empty iterator. There are many similar examples demonstrating the absence of common ground between SLs.

As a humble attempt to create such a common ground for the world of objects, we suggest our own taxonomy of objects for EO\footnote{\url{https://www.eolang.org}}, an object-oriented programming language with an intentionally reduced feature set~\citep{bugayenko2021eolang}. All objects are grouped as such:

\begin{itemize}
    \item Primitives (\cref{sec:primitives})
    \item Array (\cref{sec:array})
    \item Flow control abstractions (\cref{sec:flow})
    \item Non-standard bit-width numbers (\cref{sec:digits})
    \item Memory abstractions (\cref{sec:memory})
    \item Math functions and algorithms (\cref{sec:math})
    \item String manipulations (\cref{sec:strings})
    \item Collections (\cref{sec:collections})
    \item I/O Streams (\cref{sec:streams})
    \item File System abstractions (\cref{sec:fs})
    \item Net Sockets (\cref{sec:sockets})
    \item Synchronization in multi-threading (\cref{sec:synchronization})
\end{itemize}

An object in EO is either a composition of other objects or an atom, which is a foreign function interface to a lower-level runtime such as, for example, JVM or Assembly. In this paper, we denote atoms by a star, for example: \aff{times}.

Everywhere in the paper we use dark blue color to highlight an object when it's mentioned for the first time.

In this paper we don't provide exact specification of each object, in order to avoid conflicts with their exact definitions in Objectionary\footnote{\url{https://www.objectionary.com}}.

\section{Principles}

We adhere to a few principles:

\begin{itemize}
    \item Wherever possible, objects must be implemented in EO, instead of being atoms.
    \item There should be no functionality implemented two or more times in the entire taxonomy.
    \item Object names must not be abbreviated and must constitute an English sentence when being used through dot notation,
    for example:
\begin{ffcode}
x.times.y > z
\end{ffcode}
means ``$x$ times $y$ is $z$.''
There are a few exceptions though, like \ff{eq}, \ff{seq}, \ff{lt}, and a few others.
\end{itemize}

In case of any runtime error, an object is supposed to ``return'' an error object, with \ff{message} attribute present (as \ff{string}) and \ff{@} attribute absent.

\section{Primitives}\label{sec:primitives}

There are five primitive data objects described below. All of them have \adeff{as-bytes} attribute, which is an abstraction of their representation as a sequence of bytes.
All primitives also have \deff{as-hash} attribute, which, being an \ff{int}, is their own implementation of a hash function.
They also have \deff{eq} attribute, which is \ff{TRUE} if the primitive is equal to another object, through the use of \ff{\^{}.as-bytes.eq}.

\subsection{Bytes}

The center of the taxonomy is \deff{bytes} data object, which is an abstraction of an sequence of bytes. The sequence can either be empty or contain an theoretically unlimited number of bytes. A byte is a unit of information that consists of eight adjacent binary digits (bits), each of which consists of a 0 or 1. In EO syntax a sequence of bytes is denoted as either \eohex{--} (double dash) for an empty one or as a concatenation of \eohex{xx-}, where \ff{xx} is a hexadecimal representation of a byte.

The object \ff{bytes} has these four attributes, to turn itself into one of four primitive data objects:
\adeff{as-int},
\adeff{as-bool},
\adeff{as-float},
and
\adeff{as-string}. The attribute \aff{as-int} expects the size of the sequence to be less or equal to eight. The attribute \aff{as-bool} expects exactly one byte to be in the sequence. The attribute \aff{as-float} expects exactly eight bytes.

There are also attributes for bitwise operations:
\adeff{and},
\adeff{or},
\adeff{xor},
\adeff{left},
and
\adeff{right}.

The object also has attribute \adeff{eq} for comparing itself with another sequence of bytes.

The attribute \adeff{size} is the total number of bytes in the sequence.

The attribute \adeff{slice} represents a sub-sequence inside the current one.

The attribute \adeff{concat} is a new sequence, which is a concatenation of two byte sequences: the current and the provided one.

\subsection{Int}

The object \deff{int} is an eight-bytes abstraction of a two's complement signed integer in a big-endian order.

The attribute \adeff{plus} is a summary of this number and another \ff{int}.
The attribute \deff{minus} is a subtraction of another \ff{int} from this number.
The attributes \adeff{times} and \adeff{div} are a multiplication and a division of this number and another \ff{int}.
The attribute \deff{neg} is the same number with a different sign.

The attribute \adeff{lt} is a \ff{bool} object \ff{TRUE} if it's ``less than'' another \ff{int} and \ff{FALSE} otherwise. Attributes \deff{gt}, \deff{lte}, and \deff{gte} are ``greater than,'' ``less than or equal,'' and ``greater than or equal'' comparisons respectively.

Integers with a different bit-width are explained in \cref{sec:digits}.
All other manipulations with integers can be done in their decorators. Some of them are explained in \cref{sec:math}.

\subsection{Bool}

The object \deff{bool} is a one-bit abstraction of a Boolean value. Its \aff{as-bytes} is either \eohex{00-} or \eohex{01-}.

There are two pre-defined objects \deff{TRUE} and \deff{FALSE}.

The attribute \deff{not} is a Boolean inversion. Attributes \deff{and}, \deff{or}, \deff{xor} are logical conjunction, disjunction, and exclusive disjunction of the object itself with an a array of other Boolean values; they use \aff{and}, \aff{or}, and \aff{xor} from \ff{\^{}.as-bytes} respectively.

The attribute \adeff{if} is a branching mechanism, expecting three arguments: a condition as \ff{bool}, a ``positive'' object, and a ``negative'' object. When being dataized, the object \aff{if} first dataizes the condition and then, according to the result obtained, either dataizes the positive object or the negative one, returning the result as a result of its own dataization. Both positive and negative objects must be copies of the same abstract object. This code give a title to a random number:

\begin{ffcode}
QQ.io.stdout
  QQ.txt.sprintf
    "Coin toss: %s"
    if.
      random.gte 0.5
      "heads"
      "tail"
\end{ffcode}

The attribute \adeff{while} is an iterating mechanism, expecting two attributes: a condition as \ff{bool} and an abstract object with one free attribute. When being dataized, the object \aff{while} first dataizes the condition and then either stops if it is \ff{FALSE} or dataizes the abstract object with an \ff{int} equal to the number of the iterating cycle. This code makes a few attempts to find a random number that is smaller than 0.1:

\begin{ffcode}
while.
  random.gte 0.1
  [i]
    QQ.io.stdout > @
      QQ.txt.sprintf
        "Attempt #%d" i
\end{ffcode}

The attribute \deff{switch} is an abstraction of a multi-case branching, where the object itself is compared sequentially with $k$ in provided $(k, v)$ pairs and the dataization result is equal to $v$ of the first matching pair, while the last pair is the one to be used if no other pairs match:

\begin{ffcode}
QQ.io.stdout
  TRUE.switch
    *
      *
        password.eq "swordfish"
        "password is correct!"
      *
        password.eq ""
        "empty password is not allowed"
      *
        FALSE
        "password is wrong"
\end{ffcode}

\subsection{Float}

The object \deff{float} is an abstraction of a floating point number, which takes eight bytes and fills them up according to the format suggested by \citet{ieee754}.

The attribute \adeff{plus} is a summary of this number and another \ff{float}.
The attribute \deff{minus} is a subtraction of another \ff{float} from this number.
The attributes \adeff{times} and \adeff{div} are a multiplication and a division of this number and another \ff{float}.
The attribute \deff{neg} is the same number with a different sign.

The attribute \adeff{lt} is a \ff{bool} object \ff{TRUE} if it's ``less than'' another \ff{float} and \ff{FALSE} otherwise. Attributes \deff{gt}, \deff{lte}, and \deff{gte} are ``greater than,'' ``less than or equal,'' and ``greater than or equal'' comparisons respectively.

Floating-point numbers with a different bit-width are explained in \cref{sec:digits}.
All other manipulations with floating-point numbers can be done in their decorators. Some of them are explained in \cref{sec:math}.

\subsection{String}

The object \deff{string} is an abstraction of a piece of Unicode text in UTF-8 encoding, according to \citet{unicode}. UTF-8 uses one byte for the first 128 code points, and up to 4 bytes for other characters.

The attribute \adeff{slice} takes a piece of a string as another \ff{string}. The attribute \adeff{length} is a total count of characters in the string.

Strings in other encodings, such as UTF-16 or CP1251, may be implemented through direct manipulations with \ff{bytes}.

All other manipulations with strings are explained in \cref{sec:strings}.

\section{Array}\label{sec:array}

The object \deff{array} is an abstraction of an immutable sequence of objects. For example, this code represents a two-dimensional array \ff{x} of numbers and strings:

\begin{ffcode}
* > x
  * "green" 0x0D1
  * "red" 0xF21
  * "blue" 0x11C
\end{ffcode}

The attribute \adeff{with} is a new \ff{array} with all elements from the current array and a new element added to the end of it.

The attribute \adeff{at} is the $i$-th element of the array.

The attribute \adeff{length} is the total amount of elements in the array.

All other manipulations with arrays can be implemented in their decorators. Some of them are explained in \cref{sec:collections}.

\section{Flow Control}\label{sec:flow}

Here we define objects that help control dataization flow.

The object \adeff{seq} is an abstraction of sequence of objects to be dataized sequentially. For example, this code prints one message to the console and then terminates the program due to the inability to dataize the division by zero in the middle:

\begin{ffcode}
seq
  QQ.io.stdout "Hello, world!
  42.div 0
  QQ.io.stdout "Bye!"
\end{ffcode}

The object \adeff{goto} enables forward and backward ``jumps'' either immediately finishing dataization or restarting it. For example, this code won't print a message to the console if \ff{x} is equal to zero:

\begin{ffcode}
goto
  [g]
    seq > @
      if. > y
        x.eq 0
        g.forward 0
        42.div x
      QQ.io.stdout
        QQ.txt.sprintf "42/x = %d" y
\end{ffcode}

In this example, the number will be read from the console again, if it's equal to zero:

\begin{ffcode}
goto
  [g]
    seq > @
      as-int. > x
        QQ.txt.parsed
          QQ.io.stdin
      if.
        x.eq 0
        g.backward
        QQ.io.stdout
          QQ.txt.sprintf
            "Number %d is OK!" x
\end{ffcode}

The object \adeff{try} enables catching of dataization failures and encapsulates three abstract objects. For example, the following code prints "division by zero" and then "finally":

\begin{ffcode}
try
  []
    42.div 0 > @
  [ex]
    QQ.io.stdout > @
      ex.msg
  []
    QQ.io.stdout > @
      "finally"
\end{ffcode}

\section{Digits}\label{sec:digits}

Here we define objects that represent numbers with smaller or larger bitwise width comparing with the default 64-bits convention.
All objects presented in this Section belong to \ff{QQ.digits} package.

Objects \deff{int8}, \deff{int16}, \deff{int32}, and \deff{int128} are decorators of \ff{bytes} with a predefined size. They implement the same numeric operations as \ff{int}.
Objects \deff{float16}, \deff{float32}, and \deff{float128} are also decorators of \ff{bytes}, implementing the same operations as \ff{float}.

\section{Memory}\label{sec:memory}

Here we define objects that manage memory.

The object \adeff{memory} is an abstraction of a read-write memory capable of storing one of five primitive data objects mentioned in \cref{sec:primitives}. The attribute \adeff{write} implements writing operation, while the object \aff{memory} itself decorates the data object stored in it. This code writes a number into memory and then reads it back and prints:

\begin{ffcode}
memory > m
seq
  m.write 42
  QQ.io.stdout
    QQ.txt.sprintf "Memory contains %d" m
\end{ffcode}

The object \adeff{cage} is similar to \aff{memory}, but it can store any object, not only five primitives. The use of \aff{cage} is strongly discouraged. The following code stores a ``book'' in a \aff{cage}, then retrieves it back, and prints one of its attribute:

\begin{ffcode}
[] > book
  title > "Object Thinking"
  author > "David West"
cage > c
seq
  c.write book
  QQ.io.stdout
    QQ.txt.sprintf
      "Title is %s" c.title
\end{ffcode}

The object \deff{ram} is an abstraction of a random-access memory of certain size, which has two attributes \adeff{write} and \adeff{read} enabling writing and reading respectively. In this example, one kilobyte of memory is allocated when a copy of \aff{ram} is created, 13 bytes is written into it starting with the 200-th byte, and then only five of them read back and \ff{"Hello"} is printed:

\begin{ffcode}
ram 1024 > d
seq
  d.write 200 "Hello, world!"
  QQ.io.stdout
    as-string.
      d.read 200 5
\end{ffcode}

The object \deff{heap} encapsulates a copy of \aff{ram} and is an abstraction of a memory management as it exists in \citet{posix}. There are two attributes \deff{malloc} and \deff{free}, which allocate and free memory blocks respectively. An allocation of a memory block creates a copy of \deff{heap.pointer} object, which is a decorator of \ff{int} with two additional attributes \deff{write} and \deff{read}.
%
In this code, 13 bytes are allocated, a string is stored into them, then read and printed, and then the bytes are freed:

\begin{ffcode}
heap (ram 1024) > h
seq
  h.malloc > p
  p.write "Hello, world!"
  QQ.io.stdout
    (p.read 13).as-string
  h.free p
\end{ffcode}


\section{Math}\label{sec:math}

Here we define objects that represent math functions and algorithms.
All objects in this Section belong to \ff{QQ.math} package.

The object \adeff{random} is a abstraction of a random \ff{float} between zero and one inclusively.

The object \deff{angle} is a decorator of \ff{float}. It assumes that the angle is in radians and has the following attributes for trigonometric functions:
\deff{sine}, \deff{cosine}, and \deff{tangent}. There are attributes \deff{as-degrees} and \deff{as-radians} for switching from radians to degrees and backwards.

The object \deff{rich-float} is a decorator of \ff{float} with the following additional attributes:

\begin{itemize}
    \item \deff{abs}: absolute value of itself ($|\rho|$)
    \item \deff{ceil}: round itself up
    \item \deff{floor}: round itself down
    \item \deff{ln}: natural logarithm ($\ln \rho$)
    \item \deff{log}: logarithm ($\log_x \rho$)
    \item \deff{max}: the greater of itself and another number
    \item \deff{min}: the smaller of itself and another number
    \item \deff{power}: itself raised to the power ($\rho^x$)
    \item \deff{signum}: the sign of itself
    \item \deff{sqrt}: square root of itself ($\sqrt\rho$)
\end{itemize}

The object \deff{rich-int} is a decorator of \ff{int} and has the same attributes as \ff{rich-float}.

\section{Strings}\label{sec:strings}

All objects presented in this Section belong to \ff{QQ.txt} package.

The object \deff{parsed} encapsulates a \ff{string} and has a few attributes: \deff{as-int}, \deff{as-hex}, \deff{as-octal}, \deff{as-bytes}, \deff{as-float}, and \deff{as-bool}. For example, this code parses a hexadecimal number from the console:

\begin{ffcode}
as-hex.
  QQ.txt.parsed
    QQ.io.stdin
\end{ffcode}

The object \deff{sprintf} builds a string according to the format provided and compliant with \citet[Chapter 5]{posix}, for example:

\begin{ffcode}
QQ.txt.sprintf
  "Hi, %d! The weight is %0.2f."
  "Jeff"
  3.14
\end{ffcode}

The object \deff{text} is a decorator of a \ff{string}, it has the following attributes:

\begin{itemize}
    \item \deff{contains}: checks whether it contains $x$
    \item \deff{ends-with}: checks whether it ends with $x$
    \item \deff{join}: joins an array with this one as a glue
    \item \deff{lower-case}: makes it lower case
    \item \deff{left-trim}: removes leading spaces
    \item \deff{replace}: finds and replaces a sub-string
    \item \deff{right-trim}: removes trailing spaces
    \item \deff{split}: breaks it into an array of strings
    \item \deff{starts-with}: checks whether it starts with $x$
    \item \deff{trim}: removes both leading and trailing spaces
    \item \deff{upper-case}: makes it upper case
\end{itemize}

The object \deff{regex} is an abstraction of a regular expression, in accordance with \citet[Part~2]{posix} (Extended Regular Expressions) and full support of Unicode. The attribute \deff{match} is a split of the \ff{string} into parts:

\begin{ffcode}
QQ.txt.regex "/(и([^ ]+))/i" > r
r.match > m
  "Из искры возгорится пламя"
eq.
  m
  *
    *
      "Из"
      0
      *
        "И"
        * "з" 1 *
    * " "
    *
      "искры"
      3
      *
        "и"
        * "скры" 4 *
    * " возгор"
    *
      "ится"
      15
      *
        "и"
        "тся" 16 *
    * " пламя"
\end{ffcode}

\section{Collections}\label{sec:collections}

Here we discuss abstractions of lists, maps, sets, and other ``collections.'' All objects presented in this Section belong to \ff{QQ.collections} package.

\subsection{List}

The object \deff{list} is a decorator of \ff{array}.

The attribute \deff{is-empty} is \ff{TRUE} if the length of the array is zero.

The attribute \deff{eq} is \ff{TRUE} if each element of the array is equal to the corresponding element of another array and the lengths of both arrays are the same.

The attribute \deff{without} is a new array with the $i$-th element removed.

The attributes \deff{each}, \deff{reduced}, \deff{found}, \deff{filtered}, and \deff{mapped} are respectively similar to \ff{forEach}, \ff{reduce}, \ff{find}, \ff{reduce}, and \ff{map} methods of \ff{Array} object in JavaScript~\citep{EcmaScript}. A few ``twin'' attributes \deff{eachi}, \deff{reducedi}, \deff{filteredi}, \deff{foundi}, and \deff{mappedi} are semantically the same, but with an extra \ff{int} argument as a counter of a cycle.

The attribute \deff{slice} is a part of the array.

The attribute \deff{sorted} is a new array with elements sorted using given comparison abstract object with two free attributes.

The attribute \deff{reversed} is a new array with elements positioned in a reverse order.

\subsection{Map}

The object \deff{map} is a decorator of \ff{array} of pairs $(k, v)$. The \ff{map} ensures that all $k$ are always unique. It is expected that each $k$ has \ff{as-hash} attribute that behaves as \ff{int}.

The attributes \ff{with}, \ff{without}, \ff{found}, and \ff{foundi} are reimplemented in \ff{map}.

\subsection{Set}

The object \deff{set} is a decorator of \ff{array}. The \ff{set} ensures that all elements in the array are always unique. It is expected that each $k$ has \ff{as-hash} attribute that behaves as \ff{int}.

The attributes \ff{with}, \ff{without}, \ff{found}, and \ff{foundi} are reimplemented in \ff{set}.

\subsection{Mutable Arrays}

The objects \deff{array-list} and \deff{linked-list} are the abstractions of mutable lists that behave like \ff{array}. They have three additional attributes: \deff{inject}, \deff{replace}, and \deff{extract}. They encapsulate \aff{cage} object.

\section{I/O Streams}\label{sec:streams}

Here we define manipulations with input and output streams. All objects presented in this Section belong to \ff{QQ.io} package.

It is expected that an input stream has the following interface:

\begin{ffcode}
[] > input
  data > @
  [max-count] > read /input
  [] > close
\end{ffcode}

It is expected that an output stream has the following interface:

\begin{ffcode}
[] > output
  [bytes-to-write] > write /input
  [] > close
\end{ffcode}

The object \adeff{stdout} is an ``output'' that prints a \ff{string} to the standard output stream, while the object \adeff{stderr} is an ``output'' that prints to the standard error stream.

The object \adeff{stdin} is an abstraction of a \ff{string} currently available in the standard input stream until the EOL character (\ff{\char`\\n}) and is an ``input.'' If there is no string in the stream, the object blocks dataization and waits. This code endlessly reads strings from the console and immediately prints them back:

\begin{ffcode}
TRUE.while
  [i]
    QQ.io.stdout > @
      QQ.io.stdin
\end{ffcode}

The object \adeff{bytes-as-input} is an ``input'' made from \ff{bytes}.

The object \adeff{memory-as-output} is an ``output'' directed towards a copy of \ff{memory}.

The object \adeff{copied} is channel between an ``input'' and an ``output,'' which moves all available bytes from the former to the latter when being dataized. For example, this code write a text to a temporary file:

\begin{ffcode}
QQ.io.copied
  QQ.io.bytes-as-input
    as-bytes.
      "Hello, world!"
  as-output.
    QQ.fs.file "/tmp/foo.txt"
    "w+"
\end{ffcode}

The object \deff{content} is a \ff{string} representation of the entire content of an ``input.''

\section{File System}\label{sec:fs}

Here we define manipulations with files and directories.
All objects presented in this Section belong to \ff{QQ.fs} package.

The object \deff{file} is an abstraction of a file and is a decorator of \ff{string}, which is the path of the file.
%
The attribute \adeff{exists} is \ff{TRUE} if the file exists.
%
The attribute \adeff{is-dir} is \ff{TRUE} if the file is a directory.
%
The attribute \adeff{touch} makes sure the file exists.
%
The attribute \adeff{rm} removes the file.
%
The attribute \adeff{mv} renames or moves the file to a new place.
%
The attribute \adeff{size} is the size of the file in bytes.
%
The attribute \adeff{as-output} is an output stream for this file, while the attribute \adeff{as-input} is an input stream.

The object \deff{directory} is an abstraction of a directory.
%
The attribute \deff{mkdir} creates the directory with all its parent directories.
%
The attribute \deff{rm-rf} deletes the directory recursively with all its subdirectories.
%
The attribute \deff{walk} finds files in the directory using glob pattern matching.
%
The attribute \deff{tmpfile} is a temporary file in the directory.

The object \adeff{tmpdir} is a system \ff{directory} of temporary files.

The object \deff{path} is a decorator of a \ff{string} and is a path of a file.
%
The attribute \adeff{resolve} is a new path with a suffix appended to the current one.
%
The attribute \adeff{dir-name} is the name of the directory in the path.
%
The attribute \adeff{ext-name} is the extension in the path.
%
The attribute \adeff{base-name} is the name of the file in the path.

\section{Sockets}\label{sec:sockets}

Here we define objects used for both server-side and client-side TCP sockets in accordance with \citet{posix}.
All objects presented in this Section belong to \ff{QQ.net} package.

The object \deff{socket} is an abstraction of a TCP socket. The attribute \adeff{connect} establishes a connection and makes a copy of the socket object. Then, its attributes \adeff{as-input} and \adeff{as-output} may be used for reading from the socket and for writing to it. For example, this code opens a connection to the 80th TCP port of \ff{google.com} and then reads the entire stream as a Unicode string:

\begin{ffcode}
connect. > s
  QQ.net.socket
    "google.com"
    80
QQ.io.stdout
  QQ.io.content
    s.as-input
s.close
\end{ffcode}

The attribute \adeff{listen} make a copy of the socket and establishes a server connection. Then, the attribute \adeff{accept} can be used to wait for an incoming connection and then make a copy when it arrives. For example, this code binds to the 8080th port on a localhost and when a connection arrives, prints a message into it:

\begin{ffcode}
listen. > s
  QQ.net.socket
    "127.0.0.1"
    8080
QQ.io.copied
  QQ.io.bytes-as-input
    "Hello, world!".as-bytes
  s.accept.as-output
s.close
\end{ffcode}

The attribute \adeff{close} terminates the connection.

\section{Synchronization}\label{sec:synchronization}

Here we define objects used for explicit synchronization between threads.
All objects presented in this Section belong to \ff{QQ.sync} package.

The object \deff{semaphore} is a counting semaphore with two attributes: \adeff{capture} and \adeff{release}.

The object \deff{atomic-memory} is a synchronized decorator of \ff{memory} with a binary \ff{semaphore} encapsulated.

\printbibliography

\end{document}
